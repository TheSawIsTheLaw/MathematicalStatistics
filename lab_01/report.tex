\documentclass[a4paper,oneside,12pt]{extreport}

\usepackage{mmap}
\usepackage[T2A]{fontenc}
\usepackage[utf8]{inputenc}
\usepackage[english,russian]{babel}

\renewcommand{\ttdefault}{PTMono-TLF}

% Текст отчёта следует печатать, соблюдая следующие размеры полей:
% левое — 30 мм, правое — 15 мм, верхнее и нижнее — 20 мм.
\usepackage[left=20mm, right=15mm, top=15mm, bottom=15mm]{geometry}

% \setlength{\parindent}{1.25cm} % Абзацный отступ

\usepackage{setspace}
% \onehalfspacing % Полуторный интервал

\frenchspacing % Равномерные пробелы
\usepackage{indentfirst} % Красная строка

\usepackage{microtype}
\sloppy

\usepackage{titlesec}
\titlespacing*{\chapter}{0pt}{-30pt}{8pt}
\titlespacing*{\section}{\parindent}{*4}{*4}
\titlespacing*{\subsection}{\parindent}{*4}{*4}
\titleformat{\chapter}{\LARGE\bfseries}{\thechapter}{20pt}{\LARGE\bfseries}
\titleformat{\section}{\Large\bfseries}{\thesection}{40pt}{\Large\bfseries}

\usepackage{graphicx}
\usepackage{caption}
\usepackage{float}

\usepackage[unicode,pdftex]{hyperref}
\hypersetup{hidelinks}

%% begin title
\usepackage{wrapfig}

\makeatletter
	\def\vhrulefill#1{\leavevmode\leaders\hrule\@height#1\hfill \kern\z@}
\makeatother
%% end title

%% begin code
\usepackage{listings}
\usepackage{listingsutf8}
\usepackage{xcolor}

\lstset{language=Matlab,%
    %basicstyle=\color{red},
    breaklines=true,%
    morekeywords={matlab2tikz},
    keywordstyle=\color{blue},%
    morekeywords=[2]{1}, keywordstyle=[2]{\color{black}},
    identifierstyle=\color{black},%
    stringstyle=\color{mylilas},
    commentstyle=\color{mygreen},%
    showstringspaces=false,%without this there will be a symbol in the places where there is a space
    numbers=left,%
    numberstyle={\tiny \color{black}},% size of the numbers
    numbersep=9pt, % this defines how far the numbers are from the text
    emph=[1]{for,end,break},emphstyle=[1]\color{red}, %some words to emphasise
    %emph=[2]{word1,word2}, emphstyle=[2]{style},    
}

%% end code

%% begin theorem
\usepackage{amsthm}

\makeatletter
\newtheoremstyle{indented}
	{}% measure of space to leave above the theorem
	{}% measure of space to leave below the theorem
	{}% name of font to use in the body of the theorem
	{\parindent}% measure of space to indent
	{\bfseries}% name of head font
	{.}% punctuation between head and body
	{ }% space after theorem head; " " = normal interword space
	{}% header specification (empty for default)
\makeatother

\theoremstyle{indented}

\newtheorem{definition}{Определение}[section]
\newtheorem{remark}{Замечание}[section]
%% end theorem


\usepackage{amsmath, amsfonts, amssymb}

\renewcommand\thesection{\arabic{section}}
\renewcommand\thesubsection{\thesection.\arabic{subsection}}

\begin{document}
\paragraph{Цель работы:} построение гистограммы и эмпирической функции распределения.

\paragraph{Содержание работы}

\begin{enumerate}
	\item Для выборки объёма $n$ из генеральной совокупности $X$ реализовать в виде программы на ЭВМ
	\begin{itemize}
		\item вычисление максимального значения $M_{\max}$ и минимального значения $M_{\min}$;
		\item размаха $R$ выборки;
		\item вычисление оценок $\hat\mu$ и $S^2$ математического ожидания $MX$ и дисперсии $DX$;
		\item группировку значений выборки в $m = [\log_2 n] + 2$ интервала;
		\item построение на одной координатной плоскости гистограммы и графика функции плотности распределения вероятностей нормальной случайной величины с математическим ожиданием $\hat{\mu}$ и дисперсией $S^2$;
		\item построение на другой координатной плоскости графика эмпирической функции распределения и функции распределения нормальной случайной величины с математическим ожиданием $\hat{\mu}$ и дисперсией $S^2$.
	\end{itemize}
	\item Провести вычисления и построить графики для выборки из индивидуального варианта.
\end{enumerate}

\pagebreak
\section{Формулы для вычисления величин}

$$\vec x=(x_1, ..., x_n)$$

\begin{enumerate}
\item \textbf{Максимальное значение выборки}
$$M_{\max} = \max\{x_1, .., x_n\}$$
\item \textbf{Минимальное значение выборки}
$$M_{\min} = \min\{x_1, .., x_n\}$$
\item \textbf{Размах выборки}
$$R = M_{\max} - M_{\min}$$
\item \textbf{Выборочное среднее (математическое ожидание)}
$$\hat \mu(\vec x) = \frac{1}{n} \sum_{i=1}^n x_i$$
\item \textbf{Состоятельная оценка дисперсии}
$$S^2 (\vec x) = \frac{1}{n-1} \sum_{i=1}^n (x_i - \overline x)^2$$

где $ \overline{x} = \frac{1}{n} \sum_{i=1}^n x_i$
\end{enumerate}

\section{Определение эмпирической плотности и гистограммы}

\hfill 

    \textbf{Эмпирической плотностью} (отвечающей выборке $\vec x$) называют функцию

    \begin{equation*}
        \hat f_n(x) =
        \begin{cases}
            \frac{n_i}{n \Delta}, x \in J_i, i = \overline{1; p} \\
            0, \text{ иначе} \\
        \end{cases}
    \end{equation*}

где $(J_i, n_i)$ -- интервальный статистический ряд

\hfill

Пусть $\vec x$ -- выборка из генеральной совокупности $X$. Если объем $n$ этой выборки велик, то значения $x_i$ группируют не только в статистический ряд, но и в интервальный статистический ряд. Для этого отрезок
$J = [x_{(1)}, x_{(n)}]$ (где $x_{(1)}=\min\{x_1,..,x_n\}$, $x_{(n)}=\max\{x_1,..,x_n\}$) делят на $p$ равновеликих частей:

\begin{equation*}
    J_i = [a_i, a_{i+1}), i = \overline{1; p - 1}
\end{equation*}

\begin{equation*}
    J_{p} = [a_{p}, a_{p+1}]
\end{equation*}

$$a_i = x_{(1)} + (i-1)\cdot\Delta, i = \overline{1;p+1}$$

$$\Delta = \frac{|J|}{p} = \frac{x_{(n)} - x_{(1)}}{p}$$

    Интервальным статистическим рядом называют таблицу

    \begin{table}[H]
        \centering
        \begin{tabular}{|c|c|c|c|c|}
            \hline
            $J_1$ & ... & $J_i$ & ... & $J_p$ \\
            \hline
            $n_1$ & ... & $n_i$ & ... & $n_p$ \\
            \hline
        \end{tabular}
    \end{table}

    Здесь $n_i$ -- количество элементов выборки $\vec x$, которые
    $\in J_i$
    
    \hfill
    
    В нашем случае $p=m=[\log_2n] +2$
    
    \hfill

\textbf{Гистограммой} называют график эмпирической плотности. 

\section{Определение эмпирической функции распределения}

\hfill 

Пусть $\vec x = (x_1, ..., x_n)$ -- выборка из генеральной совокупности $X$. Обозначим $n(x, \vec x)$ -- чисор элементов вектора $\vec x$, которые имеют значения меньше $x$.

\hfill

\textbf{Эмпирической функцией распределения} называют функцию 

$F_n : \mathbb{R} \to \mathbb{R}$, определенную условием $F_n(x) = \frac{n(x, \vec x)}{n}$. 

\section{Текст программы}

\hfill 

\begin{lstlisting}

\end{lstlisting}

\section{Результаты расчетов для выборки из индивидуального варианта (вариант 22)} 

\begin{enumerate}
\item Максимальное значение выборки
\item Минимальное значение выборки
\item Размах выборки
\item Выборочное среднее (математическое ожидание)
\item Состоятельная оценка дисперсии
\item Группировка значений выборки в $m = [log_2 n] + 2$ интервала

%\includegraphics{values}

\item Гистограмма и график функции плотности распределения вероятностей нормальной случайной величины с математическим ожиданием $\hat \mu$ и дисперсией $S^2$. 

%\includegraphics{graph1}

\newpage

\item График эмпирической функции распределения и функции распределения нормальной случайной величины с математическим ожиданием $\hat \mu$ и дисперсией $S^2$. 

%\includegraphics[scale=0.9]{graph2}

\end{enumerate}
\end{document}